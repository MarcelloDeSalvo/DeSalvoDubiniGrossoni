\documentclass[table, 12pt]{article} % \documentclass{} is the first command in any LaTeX code.  It is used to define what kind of document you are creating such as an article or a book, and begins the document preamble

\usepackage[T1]{fontenc}
\usepackage[utf8]{inputenc}
\usepackage[english]{babel}
\usepackage{graphicx}
\usepackage{titlesec}
\usepackage{hyperref}
\usepackage[usenames,dvipsnames]{xcolor}
\usepackage{float}
\usepackage[export]{adjustbox}
\usepackage{longtable}
\usepackage{listings}
%\usepackage{alloy}



% The preamble ends with the command \begin{document}
\begin{document} % All begin commands must be paired with an end command somewhere
    
    \begin{titlepage}
        \centering
        {\scshape\large AY 2022/2023 \par}
        \vfill
        \includegraphics[width=200pt]{assets/logo_polimi}\par\vspace{1cm}
        \vspace{0.5cm}
        {\huge\bfseries RASD\@: Requirement Analysis
            and Specification Document \par}
        \vspace{1cm}
        {\large {Marcello De Salvo\quad Riccardo Grossoni \par Francesco Dubini}\par}
        \vfill
        {\large Professor\par
            Elisabetta \textsc{Di Nitto}}
        \vfill
        {\large \textbf{Version 0.1}\\ \today \par}
    \end{titlepage}
    
    \thispagestyle{plain}
    \pagenumbering{gobble}

    % Blank page
    \mbox{}

    % Table of content page
    \newpage
    \pagenumbering{roman}
    \tableofcontents

    % Start of the document
    \newpage
    \pagenumbering{arabic}

    \section{Introduction}
        \emph{Electric mobility} (e-Mobility) is a way to reduce the carbon footprint caused by motorized vehicles in urban and sub-urban areas.\\
        % One way to do so while using an electric vehicle is to fit and optimize the charging process into our daily schedule
        Comfortably knowing how to fit the charging process into one's daily schedule is a fundamental step towards that goal.
        
    \subsection{Purpose} %goals of the project

        %One of the biggest drawbacks of electric vehicles is the toll the charging process takes in terms of time. So on and so forth
         
        In the last 3 years the number of electric vehicles has doubled and, with the proposed european legislation on banning diesel fuel by 2035, the future seems to be full-electric. This rapid change requires a better and more efficient infrastructure to serve this ever increasing demand.

        %In these years the number of electric vehicles has been on the rise and, as it is expcted to be quadrupled by 2050, a better working infrastructure for them is needed.
        
        %Electric vehicles will quadruple in the next 5 years. As of now, having to charge your car in the middle of the day can prove to be a very daunting task

        \emph{eMall} (e-Mobility for all) aims to ease the charging process for the users through the \emph{e-Mobility Service Provider's} (eMSP) platform,
        providing all the needed end-users' services and by actively communicating with multiple \emph{Charging Point Operators'} (CPOs) \emph{Management Systems} (CPMS).   

    \subsubsection{Goals}
        \newcounter{goalCtr}
        \underline{User goals}
        \begin{itemize}
            \item  \textbf{(\stepcounter{goalCtr}G\arabic{goalCtr}) Know about the charging stations nearby, their cost, any special offer they have.}
            \item  \textbf{(\stepcounter{goalCtr}G\arabic{goalCtr}) Book a charge in a specific charging station for a certain timeframe.}\\ description
            \item  \textbf{(\stepcounter{goalCtr}G\arabic{goalCtr}) Start the charging process at a certain station.}
            \item  \textbf{(\stepcounter{goalCtr}G\arabic{goalCtr}) Notify the user when the charging process is finished.}
            \item  \textbf{(\stepcounter{goalCtr}G\arabic{goalCtr}) Pay for the obtained service.}
        \end{itemize}

    
    \subsection{Scope} %analysis of the world and shared phenomena
    %ADD: basic service and advanced functionalities
    \subsubsection*{Shared Phenomena}
        \newcounter{sharedP}
        \rowcolors{2}{red!25}{}
        \begin{center}
        \begin{tabular}{|c|p{0.70\textwidth}|c|}
            \hline
            \textbf{ID} & \textbf{Phenomenom}\\\hline\hline
            \stepcounter{sharedP}
            S\arabic{sharedP} & User registers through the application \\\hline
            \stepcounter{sharedP}
            S\arabic{sharedP} & User logs into the application \\\hline %might remove this
            \stepcounter{sharedP}
            S\arabic{sharedP} & User gets notified by the system about the status of the charge \\\hline
            \stepcounter{sharedP}
            S\arabic{sharedP} & User is suggested by the system to go and charge the vehicle, depending on the status of the battery, his daily schedule and the charging type\\\hline
            \stepcounter{sharedP}
            S\arabic{sharedP} & User is presented with special offers made available by some CPOs\\\hline
            \stepcounter{sharedP}
            S\arabic{sharedP} & User checks the availability of charging slots at nearby stations \\\hline
            \stepcounter{sharedP}
            S\arabic{sharedP} & User pays the cost of the charge through the application \\\hline
            \hline
        \end{tabular}
        \end{center}
    
    \subsubsection*{World Phenomena}
    \newcounter{worldP}
    \begin{center}
        \rowcolors{2}{red!25}{}
        \begin{tabular}{|c|p{0.70\textwidth}|c|}
            \hline
            \textbf{ID} & \textbf{Phenomenom}\\\hline\hline
            \stepcounter{worldP}
            W\arabic{worldP} & Power outage in a station nearby area\\\hline
            \stepcounter{worldP}
            W\arabic{worldP} & Physical problem in the charging station\\\hline
            \stepcounter{worldP}
            W\arabic{worldP} & Physical problem in the electric vehicle\\\hline
            \stepcounter{worldP}
            W\arabic{worldP} & Unexpected changes in the user daily schedule\\\hline
            \stepcounter{worldP}
            W\arabic{worldP} & Person other than the user is driving the car\\\hline
            \stepcounter{worldP}
            W\arabic{worldP} & Text \\\hline
            \stepcounter{worldP}
            W\arabic{worldP} & Text \\\hline
            \hline
        \end{tabular}
    \end{center}
    
    \newpage
    \subsection{Definitions, acronyms, abbreviations}
    \subsubsection*{Definitions}
        \begin{itemize}
            \item \textbf{User}: any electric car owner.
            \item \textbf{e-Mobility Service Providers}: company offering an electric vehicle charging service to drivers by providing access to multiple charging points around a geographic area.
            \item \textbf{Charging Point Operator}: charging point stations owner.
            \item \textbf{Charge Point Management System}: charging Point Operator's IT infrastructure. Handles the acquisition of energy from external Distribution System Operators and distributes it to the connected vehicles. It can also makes automatic decisions, such as the amount of energy to be used for each connected vehicle.
            \item \textbf{Distribution System Operator}: entity responsible for distributing and managing energy from the generation sources to the final consumers.
        \end{itemize}

    \subsubsection*{Acronyms}
        \begin{itemize}
            \item \textbf{EV}: Electric vehicle
            \item \textbf{eMSP}: e-Mobility Service Providers
            \item \textbf{CPO}: Charging Point Operator
            \item \textbf{CPMS}: Charge Point Management System
            \item \textbf{DSO}: (3rd party) Distribution System Operator
            \item \textbf{API}: Application Programming Interface
            \item \textbf{UML}: Unified Modeling Language
        \end{itemize}

    \newpage
    \subsection{Revision history}
        \begin{itemize}
            \item Version 0.1: Setup
            \begin{itemize}
                \item[--] Created first layout
            \end{itemize}
        \end{itemize}

    \subsection{Reference documents}
        \begin{itemize}
            \item Specification document: "Assignment RDD AY 2022-2023"
            \item Alloy documentation: https://alloytools.org/documentation.html
            \item Data on number of electric vehicles: https://www.iea.org/data-and-statistics/charts/global-electric-car-stock-2010-2021
        \end{itemize}
    
    \subsection{Document structure}
        \begin{itemize}
            \item \textbf{Section 1}: introduces the problem, describes every goal of the project and gives an analysis of the world and shared phenomena.
            \item \textbf{Section 2}: gives an overall description of the project and all the interactions that will occur between the system and the final users, including a list of possible scenarios and a description of all the actors involved. It provides also an UML class diagram that will be used as a reference point for the developers.
            \item \textbf{Section 3}: includes all the project's requirements and an in-depth description everything presented in Section 2.
            \item \textbf{Section 4}: shows the Alloy model defined for this project.
        \end{itemize}

    \newpage



    

    \section{Overall Description}

    \subsection{Product perspective}
    In this section we mainly describe some typical scenarios, all the \textit{Product Functions (\ref{product_functions})} offered by the eMall system and the\textit{UML(\ref{class_diagram})} class-diagram.

        \subsubsection{Scenarios}
        \label{scenarios}
            \begin{enumerate}
            
            \item Mister Fontana has just bought an electric vehicle and wants to create an account for the application.  
            \begin{itemize}
                \item He opens the web app and clicks the "Sign Up" button
                \item He inserts all the required information in the mandatory fields and presses the "Confirm" button
                \item The system verifies that the mail has not been used before. After passing the verification a confirmation e-mail is sent to mister Fontana's mailbox
                \item He checks his mailbox to see if he received the confimation e-mail
            \end{itemize}
                
            \item Mister Brambilla is about to finish a meeting and is about to have a 2 hour break. eMall detects that he's about to go on break, and it sends him an e-mail suggesting him to charge his car
            \begin{itemize}
                \item He receives an e-mail suggesting him to charge his car 
                \item He logs on the eMall webapp
                \item The eMall webapp suggests the closest chargers and the ones with eventual discounts
                \item Brambilla selects his preferred charging station and books it
                \item eMall prompts a link to help Brambilla get there
                \item Brambilla opens the link and heads there
            \end{itemize}

            \item Mister Fontana is on vacation and since he's new to the area he wants to know which charging stations are nearby
            \begin{itemize}
                \item He opens the web application and logs-In
                \item He navigates to the interactive map section
                \item He clicks the "current position" button
                \item He gets notified by the web application that his GPS position is not turned on
                \item He turns on the GPS and then searches all the nearby stations by simply clicking the "current position" button and viewing the map 
                
            \end{itemize}
            
            \item Miss Sala will go to the hairdresser next sunday. She knows that in the parking lot near the hairdresser there's a charging station and she wants to make sure that a spot will be available that day, so she decides to book a spot for that charging station
            \begin{itemize}
                \item She opens the web application and logs-In
                \item She selects the "Book a spot" option
                \item She compiles the requested fields with date, time of arrival and location and selects confirm
                \item The system checks if a charging spot for that date and time is still available, in that case a confirmation message is shown and the booking is saved. If no spot is still available an "unsuccesful booking" message is shown and the booking is aborted
                \item After successfully booking a spot the system sends a mail with a recap of the booking 
                \item Miss Sala after reaching the charging station on the booked timeframe unlocks the booked charging spot via the web app and can start the starting process
            \end{itemize}
            
            \item Mister Ferrari has booked a charging spot for saturday. He realizes that he has lost the keys to his car, and will not make it in time. While he searches for his keys he forgets to delete the reservation.
            \begin{itemize}
                \item The system locks the charging spot while waiting for mister Ferrari
                \item 5 minutes after the booked time, if the reservated user doesn't show up, the charging spot is unlocked and the reservation is cancelled
                \item Other users can now start using the previously reserved charging spot
            \end{itemize}

            \item Mister Rana wants to start the charging process and pay, since he's arrived at the booked station near his office
            \begin{itemize}
                \item He opens the web application and logs-In
                \item He navigates to the "booking list" section
                \item He plugs-in the charging cable to the car 
                \item He selects the right station and clicks the "start charging" button
                \item He's then asked to pay inside the web app and he pays with his credit card
                \item He finally sees that the car has started to charge
                \item When the charging ends he gets notified by the application
            \end{itemize}

            \item Mister Rossi wants to plan his trip to Rome with his electric car. He would also like to save money by selecting the cheapest charging stations along the itinerary.
            \begin{itemize}
                \item He opens the web-application and he logs-In
                \item He plans his trip inside the navigation system or the in-app map
                \item He then sorts the charging stations by the charging price and the special offers
                \item He books the cheaper ones that covers the right amount of distance from each other
            \end{itemize}
            
            \item Mister Lamborghini wants to delete his booked charge since his schedule changed and he can no longer make it there in time.
            \begin{itemize}
                \item He opens the web-application and he logs-In
                \item On the webapp he opens his "bookings list"
                \item He then selects the one he wants to delete and clicks on "cancel booking"
                \item An email of the succesful cancellation is sent and the system deletes the previous reservation
            \end{itemize}
            

            
            \end{enumerate}


        \subsubsection{Class diagram}
        \label{class_diagram}
            Text \textit{daily plan}
            footnotemark\footnotemark.
            \footnotetext{Text}

            \begin{itemize}
                \item \textbf{Type of Data 1}\\
                Text
                \item \textbf{Type of Data 2}\\
                Text
            \end{itemize}
            
            Image of the class diagram
            \begin{center}
                \begin{figure}[H]
                    \includegraphics[scale=0.45, center]{assets/logo_polimi.jpg}
                    \caption{High-level UML}
                    \label{fig: UML}
                \end{figure}
            \end{center}

            
    \subsection{Product functions}
    \label{product_functions}
    
        \subsubsection{Sign-up and shared functions}
        \begin{itemize}
            \item \textbf{Sign-up:} let the user sign-up thorugh an email and a password.
            \footnotetext{Text}
            \begin{center}
                \begin{figure}[!h]
                    \includegraphics[width=\textwidth]{assets/logo_polimi.jpg}
                    \caption{Sign Up BPMN}
                    \label{fig: singup}
                \end{figure}
            \end{center}
        \end{itemize}
        
        \subsubsection{Other function 1}
        \begin{itemize}                                 
            \item \textbf{Text:} text
        \end{itemize}
        \footnotetext{text}
        
        
    \subsection{User characteristics}
    The application has been thought for anybody that owns an electric car and wants to plan his charging process. 
    Because of this, the potential user base comprises of any electric vehicle owner with a device connected to the internet.
    The user should also be capable of interacting with a webapp
        
    \subsection{Assumptions, dependencies and constraints}
    \newcounter{assumptionCtr}
    \begin{itemize}
        \item \stepcounter{assumptionCtr}D\arabic{assumptionCtr}: ass. 1
        \item \stepcounter{assumptionCtr}D\arabic{assumptionCtr}: ass. 2
        \item \stepcounter{assumptionCtr}D\arabic{assumptionCtr}: ass. 3\footnotemark
        \item \stepcounter{assumptionCtr}D\arabic{assumptionCtr}: ass. 4\%
        \item \stepcounter{assumptionCtr}D\arabic{assumptionCtr}: ass. 5
        \item \stepcounter{assumptionCtr}D\arabic{assumptionCtr}: ass. 6
        \item \stepcounter{assumptionCtr}D\arabic{assumptionCtr}: ass. 7
        \item \stepcounter{assumptionCtr}D\arabic{assumptionCtr}: ass. 8
        \item \stepcounter{assumptionCtr}D\arabic{assumptionCtr}: ass. 9
        \item \stepcounter{assumptionCtr}D\arabic{assumptionCtr}: ass. 10
        \item \stepcounter{assumptionCtr}D\arabic{assumptionCtr}: ass. 11
        \item \stepcounter{assumptionCtr}D\arabic{assumptionCtr}: ass. 12
        \item \stepcounter{assumptionCtr}D\arabic{assumptionCtr}: ass. 13\footnotemark[\value{footnote}]
        \item \stepcounter{assumptionCtr}D\arabic{assumptionCtr}: ass. 14
    \end{itemize}
    \footnotetext{footnote}



\end{document} % This is the end of the document