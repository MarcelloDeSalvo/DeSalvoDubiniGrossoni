\documentclass[table, 12pt]{article} % \documentclass{} is the first command in any LaTeX code.  It is used to define what kind of document you are creating such as an article or a book, and begins the document preamble

\usepackage[T1]{fontenc}
\usepackage[utf8]{inputenc}
\usepackage[english]{babel}
\usepackage{graphicx}
\usepackage{titlesec}
\usepackage{hyperref}
\usepackage[usenames,dvipsnames]{xcolor}
\usepackage{float}
\usepackage[export]{adjustbox}
\usepackage{longtable}
\usepackage{listings}
%\usepackage{alloy}



% The preamble ends with the command \begin{document}
\begin{document} % All begin commands must be paired with an end command somewhere
    
    \begin{titlepage}
        \centering
        {\scshape\large AY 2022/2023 \par}
        \vfill
        \includegraphics[width=200pt]{assets/logo_polimi}\par\vspace{1cm}
        \vspace{0.5cm}
        {\huge\bfseries RASD\@: Requirement Analysis
            and Specification Document \par}
        \vspace{1cm}
        {\large {Marcello De Salvo\quad Riccardo Grossoni \par Francesco Dubini}\par}
        \vfill
        {\large Professor\par
            Elisabetta \textsc{Di Nitto}}
        \vfill
        {\large \textbf{Version 0.1}\\ \today \par}
    \end{titlepage}
    
    \thispagestyle{plain}
    \pagenumbering{gobble}

    % Blank page
    \mbox{}

    % Table of content page
    \newpage
    \pagenumbering{roman}
    \tableofcontents

    % Start of the document
    \newpage
    \pagenumbering{arabic}

    \section{Introduction}
        \emph{Intro}
    
    \subsection{Purpose} %goals of the project
        \emph{Purpose} 

    \subsubsection{Goals}
        \newcounter{goalCtr}
        \underline{Shared goals}
        \begin{enumerate}
            \item  \textbf{(\stepcounter{goalCtr}G\arabic{goalCtr}) Goal title} \\Goal 1 
        \end{enumerate}
        \underline{eMSPs}
        \begin{enumerate}
            \item \textbf{(\stepcounter{goalCtr}G\arabic{goalCtr}) Goal title}\\
            Goal text

            \item \textbf{(\stepcounter{goalCtr}G\arabic{goalCtr}) Goal title}\\
            Goal text
        \end{enumerate}
        
    
    \subsubsection{Key Performance Indexes}
        Text
        \begin{itemize}  
            \item \textbf{(K1) K1}: K1 text.
            \item \textbf{(K2) K2}: K2 text.
            \item \textbf{(K3) K3}: k3 text.
            \item \textbf{(K4) K4}: K4 text.
            \item \textbf{(K5) K5}: K5 text.
        \end{itemize}
        
        The previous KPIs are thought as indexes to understand if the application is reaching the expected goals, especially:
        
        \begin{table}[H]
            \centering
            \begin{tabular}{|l|l|}
                \hline
                \textbf{KPI} & \textbf{Goals}\\\hline
                K1 & G1, G2, G3, G4\\\hline
                K2 & G1, G2 \\\hline
                K3 & G10\\\hline   
                K4 & G5 \\\hline
                K5 & G6, G7, G8\\\hline
            \end{tabular}
        \end{table}
    
    \subsection{Scope} %analysis of the world and shared phenomena
    %ADD: basic service and advanced functionalities
    \subsubsection*{Shared Phenomena}
        \newcounter{machineP}
        \rowcolors{2}{red!25}{}
        \begin{center}
        \begin{tabular}{|c|p{0.70\textwidth}|c|}
            \hline
            \textbf{ID} & \textbf{Phenomenom}\\\hline\hline
            \stepcounter{machineP}
            M\arabic{machineP} & Text \\\hline
            \stepcounter{machineP}
            M\arabic{machineP} & Text \\\hline %might remove this
            \stepcounter{machineP}
            M\arabic{machineP} & Text \\\hline
            \stepcounter{machineP}
            M\arabic{machineP} &  Text \\\hline
            \stepcounter{machineP}
            M\arabic{machineP} &  Text \\\hline
            \hline
        \end{tabular}
        \end{center}
    
    \subsubsection*{World Phenomena}
    \newcounter{worldP}
    \begin{center}
        \rowcolors{2}{red!25}{}
        \begin{tabular}{|c|p{0.70\textwidth}|c|}
            \hline
            \textbf{ID} & \textbf{Phenomenom}\\\hline\hline
            \stepcounter{worldP}
            W\arabic{worldP} & Text \\\hline
            \stepcounter{worldP}
            W\arabic{worldP} & Text \\\hline
            \stepcounter{worldP}
            W\arabic{worldP} & Text \\\hline
            \stepcounter{worldP}
            W\arabic{worldP} & Text \\\hline
            \stepcounter{worldP}
            W\arabic{worldP} & Text \\\hline
            \stepcounter{worldP}
            W\arabic{worldP} & Text \\\hline
            \stepcounter{worldP}
            W\arabic{worldP} & Text \\\hline
            \stepcounter{worldP}
            W\arabic{worldP} & Text \\\hline
            \stepcounter{worldP}
            W\arabic{worldP} & Text \\\hline
            \stepcounter{worldP}
            W\arabic{worldP} & Text \\\hline
            \stepcounter{worldP}
            W\arabic{worldP} & Text \\\hline
            \stepcounter{worldP}
            W\arabic{worldP} & Text \\\hline
            \stepcounter{worldP}
            W\arabic{worldP} & Text \\\hline
            \stepcounter{worldP}
            W\arabic{worldP} & Text \\\hline
            \stepcounter{worldP}
            W\arabic{worldP} & Text \\\hline
            \stepcounter{worldP}
            W\arabic{worldP} & Text \\\hline
            \hline
        \end{tabular}
    \end{center}
    
    \subsection{Definitions, acronyms, abbreviations}
    \subsubsection*{Definitions}
        \begin{itemize}
            \item \textbf{Def 1}: def 1 
            \item \textbf{Def 2}: def 2
            \item \textbf{Def 3}: def 3
            \item \textbf{Def 4}: def 4
            \item \textbf{Def 5}: def 5
            \item \textbf{Def 6}: def 6
            \item \textbf{Def 7}: def 7
            \item \textbf{Def 8}: def 8  
        \end{itemize}

    \subsubsection*{Acronyms}
        \begin{itemize}
            \item \textbf{UML}: Unified Modeling Language
        \end{itemize}

    \subsection{Revision history}
        \begin{itemize}
            \item Version 0.1: Setup
            \begin{itemize}
                \item[--] Created first layout
            \end{itemize}
        \end{itemize}

    \subsection{Reference documents}
        \begin{itemize}
            \item Specification document: "Assignment RDD AY 2022-2023"
            \item Alloy documentation: https://alloytools.org/documentation.html
        \end{itemize}
    
    \subsection{Document structure}
        \begin{itemize}
            \item \textbf{Section 1} Intro.
            \item \textbf{Section 2} Overall Description
            \item \textbf{Section 3} Section 2 Expanded + Requirements.
            \item \textbf{Section 4} Alloy.
        \end{itemize}

    \newpage



\end{document} % This is the end of the document